\documentclass{amsart}

\usepackage{fouche}

\title{a thing about things}
\author{
    Ivan Di Liberti%
  , Leonardo Larizza%
  , Fosco Loregian%
}

\usepackage{booktabs}

\def\A{\protect{\{\begin{smallmatrix}
  \text{monadi}\\
  \text{lax idem}
\end{smallmatrix}\}}}
\def\B{\protect{\{\begin{smallmatrix}
  \text{LOFS}\\
  \text{riflessivi}
\end{smallmatrix}\} }}
\def\C{\protect{\{\begin{smallmatrix}
  \text{subcat}\\
  \text{lax refl}
\end{smallmatrix} \} }}


\newcommand{\dolp}[1]{\mathrel{
  \begin{tikzpicture}[scale=#1]
    \draw[rounded corners=.25pt, line width=.4pt] (0,0) rectangle (1,1);
    \draw[line width=.3pt] (0,0) -- (1,1);
    \fill (0,0) rectangle (.4,.4);
    \end{tikzpicture}
}}

\newcommand{\lp}{
\mathchoice{\dolp{.165}}
  {\dolp{.165}}
  {\mathrel{\dolp{.1325}}}
  {\mathrel{\dolp{.1}}}
}

\begin{document}
\maketitle
Una lista di cose da fare:
\begin{itemize}
  \item rifare i FS lax (seguendo Freyd-Kelly) e partendo dalle proprietà di base ($\clE\cap \clM = ?$ Quanto è unica una fattorizzazione?)
  \item Rifare [CHK] in versione lax: 
  \[
  \xymatrix{
    \A \ar@<5pt>[dr]^1 && \ar[ll]_{3} \B \ar@<5pt>[dl]^{2'} \\
     & \C \ar@<5pt>[ur]^2 \ar@<5pt>[ul]^{1'} &
  }  
  \]  
  Per mostrare queste varie implicazioni bisogna controllare che
  \begin{itemize}
  \item[1.] Le algebre di una KZ sono una categoria lax reflective
  \item[1'.] Una subcat lax reflective dà luogo a una monade lax idem
  \item[3.] dovrebbe già stare in qualche paper
  \item[2.] $\clB\subset\clA$ subrefl implica che $({}^\perp(\hom \clB), ({}^\perp(\hom\clB))^\perp)$ è un pre-LOFS
  \item[2'.] La subcat $\{X \mid X \to * \in \clM\}$ è lax reflective
  \end{itemize}
\end{itemize}
\hrulefill

\section{Un po' di cose da fare}
\begin{definition}[Lax e oplax ortogonalità]
  
\end{definition}
\begin{remark}
  Connessione tra le due: la lax ortogonalità è oplax ortogonalità in $\clK^\co$.
\end{remark}
\begin{definition}
  La connessione di Galois tra le ortogonalità:
  \[
  {}^{\lp}(\firstblank) \dashv (\firstblank)^{\lp}
  \]
\end{definition}
\begin{definition}
  Lax prefactorization system
\end{definition}
\begin{definition}
  Left and right generated lax prefactorization system
\end{definition}
\begin{definition}
  Lax ortogonalità in una slice categoria
\end{definition}
\begin{lemma}
  TFAE:
  \begin{itemize}
    \item $f\lp f$
    \item $f \lp All$
    \item $All \lp f$
    \item $f$ è un aggiunto sinistro
  \end{itemize}
\end{lemma}
\begin{remark}
  Max e min elementi nel poset delle prefactorizations
\end{remark}
\begin{proposition}
  Proprietà di chiusura:
\begin{table}[]
  \begin{tabular}{@{}cc@{}}
  \toprule
  $\clH^{\lp}$                 & ${}^{\lp}\clH$                  \\ \midrule
  contiene tutti gli iso       & contiene tutti gli iso          \\
  chiuso per composizione      & chiuso per composizione         \\
  chiuso per push              & chiuso per pull                 \\
  chiuso per retratti          & chiuso per retratti             \\
  chiuso per compo transfinite & chiuso per op-compo transfinite \\ \bottomrule
  \end{tabular}
  \end{table}
\end{proposition}
\begin{proposition}
  Unicità della fattorizzazione: quanto è unica?
\end{proposition}
\begin{proposition}
  3-per-2 e proprietà di cancellazione:
  \begin{itemize}
    \item $\clE$ è R32; $vu,u\in\clE\To v \in\clE$;
    \item $\clM$ è L32; $v, vu\in\clM\To u\in\clM$.
  \end{itemize}
  In un PFS riflessivo, $\clM$ è 3-per-2; in un PFS coriflessivo, $\clE$ è 3-per-2.
\end{proposition}
\begin{theorem}
  Se $\bfC$ è una 2-categoria con un oggetto terminale, c'è un'aggiunzione
  \[
  \Phi : Rex(\bfC)_{lax} \leftrightarrows PF(\bfC)_{lax, \tau} : \Psi  
  \]
  dove 
  \begin{itemize}
    \item $\Psi : (\clE,\clM)\mapsto \{X \mid (X\to *)\in\clM\} \in ReX(\bfC)_{lax}$
    \item $\Phi : (\bfB\subseteq \bfC) \mapsto \big({}^{\lp}(\hom\bfB), ({}^{\lp}(\hom\bfB))^{\lp}\big)$
  \end{itemize}
\end{theorem}
\section{Alcune minchiate e conti}
Vogliamo rifare il teorema centrale di CHK succitato.

Per farlo, bisogna rifare un po' tutta la teoria dei FS lax; la corrispondenza tra monadi e subcat è già stata fatta da Adamek (chiamo così l'unico paper dove Adamek è un autore). La regola è
\[
\bfB \mapsto \Big({}^{\lp}(\hom\bfB), ({}^{\lp}(\hom\bfB))^{\lp}\Big)
\]

Dobbiamo tornare indietro, associando una subcat KZ-riflessiva a un LOFS riflessivo (cioè $\clM$ è 2-su-3). Dato un tale LOFS $(\clE,\clM)$ consideriamo la sottocategoria degli oggetti la cui freccia terminale sta in $\clM$; questo è il candidato a essere riflessivo in $\bfC$. Il trucco di [CHK] è sfruttare ortogonalità forte e fattorizzabilità (almeno dei morfismi terminali) per ottenere una riflessione. 

Il mantra: la $\clE$-parte di $X \to 1$ è l'unità della monade.

Definiamo allora $\clM/1 = \{X\mid X\to 1\in\clM\}$ e proviamo a costruire un 2-funtore $R : \bfC \to \clM/1$ che generi una KZ-monade, ossia dimostriamo che $\clM/1$ è KZ-riflessiva.

Nella definizione di Ivan, ciò significa che 
\begin{itemize}
  \item $\clM/1$ è possibilmente non-full in $\bfC$;
  \item $i : \clM/1 \hookrightarrow \bfC$ è tale che ciascun $i_{XY} : \clM/1(A,B) \to \bfC(iA,iB)$ è una riflessione. 
\end{itemize}
Per prima cosa allora dobbiamo costruire $R$, determinarne l'azione sulle 1- e  2-celle, e capire se effettivamente la composizione $iR$ è una KZ-monade. La condizione di KZ-monade è
\[\eta_{RA}\dashv \mu_A \dashv R\eta_A\]
(non è la condizione più economica da dimostrare ma è quella che costringe a esplicitare tutti i dati).
\begin{itemize}
  \item $\eta_{RA}$ è la $\clE$-parte della fattorizzazione di $RA\to 1$; si ha 
  \[RA \xto{\eta_{RA}} RRA \to 1.\]
  \item per determinare $R\eta_A$ bisogna capire come agisce $R$ sui morfismi; classicamente, dato $f : A \to B$ si usa l'ortogonalità per costruire $RA \to RB$ come unico filler di 
  \[
  \xymatrix{
    A \ar[r]^f \ar[d]_{\eta_A} & B \ar[r]^{\eta_B} & RB \ar[d]\\
    RA \ar[rr] \ar[urr]&& 1
  }  
  \] 
  Nulla cambia molto ora, tranne per il fatto che i triangoli non commutano più ma sono universali:
  \[
  \xymatrix{
    A\ar@{}[drr]|(.3)\Swarrow\ar@{}[drr]|(.7)\Swarrow \ar[r]^f \ar[d]_{\eta_A} & B \ar[r]^{\eta_B} & RB \ar[d]\\
    RA \ar[rr] \ar[urr]&& 1
  }  
  \] 
  (il triangolo superiore è una Lan, quello inferiore un Rift, e le celle ne sono rispettivamente unità e counità.)

  L'azione di $R$ sui morfismi è perciò data da una formula piuttosto familiare:
  \[Rf = \Lan_{\eta_A}(\eta_B f)\]
  Notare che il triangolo di Rift ora non dà particolari informazioni: la cella è invertibile, perché deve coincidere con l'unico isomorfismo che connette ogni coppia di 1-celle $RA\to 1$. La riflessività di $\clM$ però implica che per ogni $f : A \to B$ si abbia $Rf \in\clM$; questo corrobora l'intuizione che $Rf$ sia una sorta di $\bsP_! f$ perché in $\clM$ moralmente devono stare i morfismi cocontinui. 
  
  Altro fatto curioso: supponiamo che $f$ stia in $\clE$ e che anche $\clE$ sia 2-su-3 (ipotesi di lavoro costante quando facevo t-strutture): siccome $\eta_A, \eta_B \in\clE$ per costruzione, si ha che $Rf \in\clE\cap\clM$, cioè è un \emph{aggiunto sinistro}; sapevamo che era cocontinuo tra categorie di prefasci, e Freyd ci dava un aggiunto gratis; però è curioso che (probabilmente) una forma del teorema del funtore aggiunto sia un risultato di ortogonalità: un cocontinuo, che in più sta in $\clE$, deve essere un aggiunto sinistro.
  \item La moltiplicazione $\mu_A$ si determina a partire dall'aggiunzione con la formula $\mu = i\epsilon R$, ma ce n'è una caratterizzazione alternativa. In questo momento non è determinante trovarla (abbiamo problemi più fondamentali); però a un certo punto sarà necessario, perché qui $R$ non è idempotente. Prendere cioè la fattorizzazione di $A\to 1$ due volte non dà $RA$ come risultato.

  Quello che succede è però ben più interessante: classicamente, ciò che si fa è fattorizzare $RA\to 1$ e ottenere
  \[RA \xto{\eta_{RA}} RRA \xto{m} 1 \]
  Per 2-su-3, $\eta_{RA}\in\clE\cap\clM$, è un isomorfismo, e quindi anche $\mu$, che ce l'ha come inverso destro, deve essere un isomorfismo, quindi la monade è idempotente.

  Ora, la lax idempotenza si apprezza tutta nel fatto che $\clE\cap \clM$ è fatta dagli aggiunti sinistri! Ciò che otteniamo, ora, è infatti che $\eta_{RA}$ è un aggiunto sinistro, e il suo aggiunto destro è un candidato a fare da $\mu_A : RRA \to RA$.
\end{itemize}
\section{Un problema concreto: quanto è unica una fattorizzazione?}
Un problema che è passato sotto silenzio è questo: dobbiamo ancora capire come si risponde quindi ogni proposta è bene accetta, anche se sembra stupida.
\begin{itemize}
  \item Quanto commuta il triangolo in cui un LOFS fattorizza un morfismo? La domanda sembra stupida, ma considera l'inviluppo di Isbell di una categoria $A$: si tratta della categoria delle lax-fattorizzazioni di $\hom_A$ attraverso l'oggetto terminale. Quindi in almeno un contesto una fattorizzazione lax ``esiste in natura''.
  \item Quanto unica è la fattorizzazione una volta trovata? Questo apre una porta molto interessante... classicamente infatti, il trucco per mostrare che in una coppia di fattorizzazioni $A \to Ff \to B$ e $A \to Gf\to B$ i due oggetti $Ff,Gf$ sono isomorfi usa l'ortogonalità, e l'unicità dei lifting. Entrambe proprietà che qui abbiamo opportunamente laxificato.
\end{itemize}
\end{document}