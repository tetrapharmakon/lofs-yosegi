\documentclass{amsart}

\usepackage{fouche}

\title{a thing about things}
\author{
    Ivan Di Liberti
  , Leonardo Larizza
  , Fosco Loregian
}

\def\A{\protect{\{\begin{smallmatrix}
  \text{monadi}\\
  \text{lax idem}
\end{smallmatrix}\}}}
\def\B{\protect{\{\begin{smallmatrix}
  \text{LOFS}\\
  \text{riflessivi}
\end{smallmatrix}\} }}
\def\C{\protect{\{\begin{smallmatrix}
  \text{subcat}\\
  \text{lax refl}
\end{smallmatrix} \} }}
\begin{document}
\maketitle
Una lista di cose da fare:
\begin{itemize}
  \item rifare i FS lax (seguendo Freyd-Kelly) e partendo dalle proprietà di base ($\clE\cap \clM = ?$ Quanto è unica una fattorizzazione?)
  \item Rifare [CHK] in versione lax: 
  \[
  \xymatrix{
    \A \ar@<5pt>[dr]^1 && \ar[ll]_{3} \B \ar@<5pt>[dl]^{2'} \\
     & \C \ar@<5pt>[ur]^2 \ar@<5pt>[ul]^{1'} &
  }  
  \]  
  Per mostrare queste varie implicazioni bisogna controllare che
  \begin{itemize}
  \item[1.] Le algebre di una KZ sono una categoria lax reflective
  \item[1'.] Una subcat lax reflective dà luogo a una monade lax idem
  \item[3.] dovrebbe già stare in qualche paper
  \item[2.] $\clB\subset\clA$ subrefl implica che $({}^\perp(\hom \clB), ({}^\perp(\hom\clB))^\perp)$ è un pre-LOFS
  \item[2'.] La subcat $\{X \mid X \to * \in \clM\}$ è lax reflective
  \end{itemize}
\end{itemize}
\end{document}