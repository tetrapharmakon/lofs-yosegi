\documentclass{amsart}

\usepackage{fouche}

\title{a thing about things}
\author{
    Ivan Di Liberti%
  , Leonardo Larizza%
  , Fosco Loregian%
}

\usepackage{booktabs}

\def\A{\protect{\{\begin{smallmatrix}
  \text{monadi}\\
  \text{lax idem}
\end{smallmatrix}\}}}
\def\B{\protect{\{\begin{smallmatrix}
  \text{LOFS}\\
  \text{riflessivi}
\end{smallmatrix}\} }}
\def\C{\protect{\{\begin{smallmatrix}
  \text{subcat}\\
  \text{lax refl}
\end{smallmatrix} \} }}


\newcommand{\dolp}[1]{\mathrel{
  \begin{tikzpicture}[scale=#1]
    \draw[rounded corners=.25pt, line width=.4pt] (0,0) rectangle (1,1);
    \draw[line width=.3pt] (0,0) -- (1,1);
    \fill (0,0) rectangle (.4,.4);
    \end{tikzpicture}
}}

\newcommand{\doolp}[1]{\mathrel{
  \begin{tikzpicture}[scale=#1]
    \draw[rounded corners=.25pt, line width=.4pt] (0,0) rectangle (1,1);
    \draw[line width=.3pt] (0,0) -- (1,1);
    \fill (.6,.6) rectangle (1,1);
    \end{tikzpicture}
}}


\newcommand{\lp}{
\mathchoice{\dolp{.165}}
  {\dolp{.165}}
  {\mathrel{\dolp{.1325}}}
  {\mathrel{\dolp{.1}}}
}

\newcommand{\olp}{
\mathchoice{\doolp{.165}}
  {\doolp{.165}}
  {\mathrel{\doolp{.1325}}}
  {\mathrel{\doolp{.1}}}
}

\begin{document}
\maketitle
Una lista di cose da fare:
\begin{itemize}
  \item rifare i FS lax (seguendo Freyd-Kelly) e partendo dalle proprietà di base ($\clE\cap \clM = ?$ Quanto è unica una fattorizzazione?)
  \item Rifare [CHK] in versione lax: 
  \[
  \xymatrix{
    \A \ar@<5pt>[dr]^1 && \ar[ll]_{3} \B \ar@<5pt>[dl]^{2'} \\
     & \C \ar@<5pt>[ur]^2 \ar@<5pt>[ul]^{1'} &
  }  
  \]  
  Per mostrare queste varie implicazioni bisogna controllare che
  \begin{itemize}
  \item[1.] Le algebre di una KZ sono una categoria lax reflective
  \item[1'.] Una subcat lax reflective dà luogo a una monade lax idem
  \item[3.] dovrebbe già stare in qualche paper
  \item[2.] $\clB\subset\clA$ subrefl implica che $({}^\perp(\hom \clB), ({}^\perp(\hom\clB))^\perp)$ è un pre-LOFS
  \item[2'.] La subcat $\{X \mid X \to * \in \clM\}$ è lax reflective
  \end{itemize}
\end{itemize}
\hrulefill

\section{Un po' di cose da fare}
\begin{definition}[Lax and oplax orthogonality]
  Let $\clK$ be a 2-category. 
  \begin{itemize}
    \item If $f : A\to B, g : X\to Y$ are 1-cells in a lax-commutative square
  \[
  \xymatrix{
    A \ar@{}[dr]|\Swarrow\ar[d]_f \ar[r]^u & X\ar[d]^g \\
    B \ar[r]_v & Y
  }  
  \]
we say that \emph{$f$ is left lax-orthogonal to $g$} (or that $g$ is right lax-orthogonal to $f$) and we denote it as $f \lp g$, if there exists a 1-cell $a : B \to X$ and two 2-cells that exhibit $a$ at the same time as the left extension of $u$ along $f$, and the right lifting of $v$ along $g$:
\[
\xymatrix{
    A \ar@{}[dr]|(.3)\Swarrow
      \ar@{}[dr]|(.7)\Swarrow
      \ar[d]_f \ar[r]^u & X\ar[d]^g \\
    B \ar[ur]\ar[r]_v & Y
}
\]
\item Dually, in a oplax commutative square like 
\[
  \xymatrix{
    A \ar@{}[dr]|\Nearrow\ar[d]_f \ar[r]^u & X\ar[d]^g \\
    B \ar[r]_v & Y
  }  
  \]
we say that \emph{$f$ is left oplax-orthogonal to $g$} (or that $g$ is right oplax-orthogonal to $f$) and we denote it as $f \olp g$, if there exists a 1-cell $a : B \to X$ and two 2-cells that exhibit $a$ at the same time as the right extension of $u$ along $f$, and the left lifting of $v$ along $g$:
\[
\xymatrix{
    A \ar@{}[dr]|(.3)\Nearrow
      \ar@{}[dr]|(.7)\Nearrow
      \ar[d]_f \ar[r]^u & X\ar[d]^g \\
    B \ar[ur]\ar[r]_v & Y
}
\]
  \end{itemize}
\end{definition}
\begin{remark}
  Oplax orthogonality in $\clK$ is exactly lax orthogonality in $\clK^\co$, and viceversa, lax orthogonality in $\clK$ is exactly oplax orthogonality in $\clK^\co$.
\end{remark}
\begin{definition}
  Given a class of arrows $\clA\subseteq \hom(\clK)$, we can define 
  \begin{align}
    {}^{\lp}\clA &= \{g\in\hom(\clK)\mid g\lp a\, \forall a \in\clA\}\\
    \clA^{\lp} &= \{g\in\hom(\clK)\mid a\lp g\, \forall a \in\clA\}
  \end{align}
  This sets up the (contravariant) \emph{Galois connection of lax orthogonality}:
  \[
  {}^{\lp}(\firstblank) \dashv (\firstblank)^{\lp}
  \]
  from whose zig-zag identities we find that for two classes $\clA,\clB\subseteq\hom(\clK)$, one has $\clA \subseteq {}^{\lp}\clB$ iff $\clB \subseteq \clA^{\lp}$.
  \todo[inline]{ahem, not sure}
\end{definition}
\begin{definition}
  A \emph{lax prefactorization system} (lax PFS) is a pair of classes of 1-cells $(\clE,\clM)$ such that $\clE = {}^{\lp}\clM$ and $\clM = \clE^{\lp}$.
\end{definition}
Each of the two classes of a lax PFS determines the other by the lax orthogonality relation, so if $(\clE,\clM)$ and $(\clE,\clM')$ are two lax PFSs, then $\clE'=\clE$, and similarly for the right classes.

The class of all lax PFS on the same 2-category $\clK$ is canonically ordered by inclusion of the right classes, or reverse inclusion of the left classes:
\[(\clE,\clM)\preceq (\clE', \clM') \iff \clM \subseteq \clM' \iff \clE'\subseteq \clE.\]

The two trivial lax prefactorization systems on $\clK$ are given by the classes 
\[(\text{LAdj}(\clK), \hom(\clK)) \qquad (\hom(\clK), \text{LAdj}(\clK))\]
These are, respectively, the maximal and minimal elements in the poset of lax PFS on $\clK$.
\begin{definition}
  Given a class of 1-cells $\clA \subseteq \hom(\clK)$ we define
  \begin{itemize}
    \item the lax PFS \emph{right generated} by $\clA$ as $({}^{\lp}\clA, ({}^{\lp}\clA)^{\lp})$;
    \item the lax PFS \emph{left generated} by $\clA$ as $({}^{\lp}(\clA^{\lp}), \clA^{\lp})$.
  \end{itemize}
\end{definition}
\begin{definition}
  Lax ortogonalità in una slice categoria
\end{definition}
\begin{lemma}
  Let $f : A \to B$ a 1-cell in $\clK$; then, the following conditions are equivalent:
  \begin{enumtag}{hc}
    \item $f\lp f$;
    \item $f \lp \hom(\clK)$;
    \item $\hom(\clK) \lp f$
    \item $f$ is a left adjoint in $\clK$.
  \end{enumtag}
\end{lemma}
\begin{proof}
  It is evident that 2 and 3 both imply 1; 1 implies 4, because the problem
  \[
  \xymatrix{
    A \ar[d]_f \ar@{=}[r]& A\ar[d]^f  \\
    B \ar@{.>}[ur]_u \ar@{=}[r]& B
  }  
  \]
  has a solution $u : B \to A$ such that a 2-cell $\eta : 1_A \To uf$ exhibits a left extension and a 2-cell $\epsilon : fu \To 1$ exhibits a right lifting; each of these two conditions is equivalent to $f \dashv u$.

  Viceversa, if 
  \[
  \xymatrix{
    A \ar[d]_f \ar[r]^a & A\ar[d]^f  \\
    B \ar@{.>}[ur]_u \ar[r]_b & B
  }  
  \]
  is a problem, we can paste it to another square on the left, 
  \[
  \xymatrix{
    A \ar[d]_f \ar@{=}[r]& A\ar[d]^f \ar[r]^a  & A \ar[d]\\
    B \ar[ur]_u \ar@{=}[r]& B \ar[r]_b & B
  }  
  \]
  Since $f\dashv u$, the extension is absolute, thus repected by $a$; so $\langle au, a * \eta\rangle$ is a left extension. Similarly, the right lifting $u\cong\rift_f 1_B$ is absolute, thus respecte by $b$; so $\langle au,a * \epsilon\rangle$ is a right lifting.

  It is similarly easy to show that 1 implies 2 and 3, so we are done.
\end{proof}
\begin{proposition}
  Proprietà di chiusura:
\begin{center}
  \begin{tabular}{@{}cc@{}}
  \toprule
  $\clH^{\lp}$                 & ${}^{\lp}\clH$                  \\ \midrule
  contiene tutti gli aggiunti sx       & contiene tutti gli aggiunti sx          \\
  chiuso per composizione      & chiuso per composizione         \\
  chiuso per push              & chiuso per pull                 \\
  chiuso per retratti          & chiuso per retratti             \\
  chiuso per compo transfinite & chiuso per op-compo transfinite \\ 
  chiuso per limiti & chiuso per colimiti \\ \bottomrule
  \end{tabular}
\end{center}
\end{proposition}
\begin{proposition}
  Let $f : A \to B$ be a 1-cell in $\clK$, and $(\clE,\clM)$ a lax PFS on $\clK$; assume that there is a factorization $$A \xto{e} H\xto{m} B$$ of $f$ in two parts such that $e\in\clE, m\in\clM$. Then, the object $H$ is unique up to Frobenius adjunction, i.e. every other $H'$ such that there is a factorization $$A \xto{e'} H' \xto{m'} B$$ is connected to $H$ by an adjoint triple $$v\dashv u\dashv v : H \leftrightarrows H'.$$
\end{proposition}
\begin{proposition}
  3-per-2 e proprietà di cancellazione:
  \begin{itemize}
    \item $\clE$ è R32; $vu,u\in\clE\To v \in\clE$;
    \item $\clM$ è L32; $v, vu\in\clM\To u\in\clM$.
  \end{itemize}
  In un PFS riflessivo, $\clM$ è 3-per-2; in un PFS coriflessivo, $\clE$ è 3-per-2.
\end{proposition}
\begin{theorem}
  Se $\bfC$ è una 2-categoria con un oggetto terminale, c'è un'aggiunzione
  \[
  \Phi : Rex(\bfC)_{lax} \leftrightarrows PF(\bfC)_{lax, \tau} : \Psi  
  \]
  dove 
  \begin{itemize}
    \item $\Psi : (\clE,\clM)\mapsto \{X \mid (X\to *)\in\clM\} \in ReX(\bfC)_{lax}$
    \item $\Phi : (\bfB\subseteq \bfC) \mapsto \big({}^{\lp}(\hom\bfB), ({}^{\lp}(\hom\bfB))^{\lp}\big)$
  \end{itemize}
\end{theorem}
\section{Alcune minchiate e conti}
Vogliamo rifare il teorema centrale di CHK succitato.

Per farlo, bisogna rifare un po' tutta la teoria dei FS lax; la corrispondenza tra monadi e subcat è già stata fatta da Adamek (chiamo così l'unico paper dove Adamek è un autore). La regola è
\[
\bfB \mapsto \Big({}^{\lp}(\hom\bfB), ({}^{\lp}(\hom\bfB))^{\lp}\Big)
\]

Dobbiamo tornare indietro, associando una subcat KZ-riflessiva a un LOFS riflessivo (cioè $\clM$ è 2-su-3). Dato un tale LOFS $(\clE,\clM)$ consideriamo la sottocategoria degli oggetti la cui freccia terminale sta in $\clM$; questo è il candidato a essere riflessivo in $\bfC$. Il trucco di [CHK] è sfruttare ortogonalità forte e fattorizzabilità (almeno dei morfismi terminali) per ottenere una riflessione. 

Il mantra: la $\clE$-parte di $X \to 1$ è l'unità della monade.

Definiamo allora $\clM/1 = \{X\mid X\to 1\in\clM\}$ e proviamo a costruire un 2-funtore $R : \bfC \to \clM/1$ che generi una KZ-monade, ossia dimostriamo che $\clM/1$ è KZ-riflessiva.

Nella definizione di Ivan, ciò significa che 
\begin{itemize}
  \item $\clM/1$ è possibilmente non-full in $\bfC$;
  \item $i : \clM/1 \hookrightarrow \bfC$ è tale che ciascun $i_{XY} : \clM/1(A,B) \to \bfC(iA,iB)$ è una riflessione. 
\end{itemize}
Per prima cosa allora dobbiamo costruire $R$, determinarne l'azione sulle 1- e  2-celle, e capire se effettivamente la composizione $iR$ è una KZ-monade. La condizione di KZ-monade è
\[\eta_{RA}\dashv \mu_A \dashv R\eta_A\]
(non è la condizione più economica da dimostrare ma è quella che costringe a esplicitare tutti i dati).
\begin{itemize}
  \item $\eta_{RA}$ è la $\clE$-parte della fattorizzazione di $RA\to 1$; si ha 
  \[RA \xto{\eta_{RA}} RRA \to 1.\]
  \item per determinare $R\eta_A$ bisogna capire come agisce $R$ sui morfismi; classicamente, dato $f : A \to B$ si usa l'ortogonalità per costruire $RA \to RB$ come unico filler di 
  \[
  \xymatrix{
    A \ar[r]^f \ar[d]_{\eta_A} & B \ar[r]^{\eta_B} & RB \ar[d]\\
    RA \ar[rr] \ar[urr]&& 1
  }  
  \] 
  Nulla cambia molto ora, tranne per il fatto che i triangoli non commutano più ma sono universali:
  \[
  \xymatrix{
    A\ar@{}[drr]|(.3)\Swarrow\ar@{}[drr]|(.7)\Swarrow \ar[r]^f \ar[d]_{\eta_A} & B \ar[r]^{\eta_B} & RB \ar[d]\\
    RA \ar[rr] \ar[urr]&& 1
  }  
  \] 
  (il triangolo superiore è una Lan, quello inferiore un Rift, e le celle ne sono rispettivamente unità e counità.)

  L'azione di $R$ sui morfismi è perciò data da una formula piuttosto familiare:
  \[Rf = \Lan_{\eta_A}(\eta_B f)\]
  Notare che il triangolo di Rift ora non dà particolari informazioni: la cella è invertibile, perché deve coincidere con l'unico isomorfismo che connette ogni coppia di 1-celle $RA\to 1$. La riflessività di $\clM$ però implica che per ogni $f : A \to B$ si abbia $Rf \in\clM$; questo corrobora l'intuizione che $Rf$ sia una sorta di $\bsP_! f$ perché in $\clM$ moralmente devono stare i morfismi cocontinui. 
  
  Altro fatto curioso: supponiamo che $f$ stia in $\clE$ e che anche $\clE$ sia 2-su-3 (ipotesi di lavoro costante quando facevo t-strutture): siccome $\eta_A, \eta_B \in\clE$ per costruzione, si ha che $Rf \in\clE\cap\clM$, cioè è un \emph{aggiunto sinistro}; sapevamo che era cocontinuo tra categorie di prefasci, e Freyd ci dava un aggiunto gratis; però è curioso che (probabilmente) una forma del teorema del funtore aggiunto sia un risultato di ortogonalità: un cocontinuo, che in più sta in $\clE$, deve essere un aggiunto sinistro.
  \item La moltiplicazione $\mu_A$ si determina a partire dall'aggiunzione con la formula $\mu = i\epsilon R$, ma ce n'è una caratterizzazione alternativa. In questo momento non è determinante trovarla (abbiamo problemi più fondamentali); però a un certo punto sarà necessario, perché qui $R$ non è idempotente. Prendere cioè la fattorizzazione di $A\to 1$ due volte non dà $RA$ come risultato.

  Quello che succede è però ben più interessante: classicamente, ciò che si fa è fattorizzare $RA\to 1$ e ottenere
  \[RA \xto{\eta_{RA}} RRA \xto{m} 1 \]
  Per 2-su-3, $\eta_{RA}\in\clE\cap\clM$, è un isomorfismo, e quindi anche $\mu$, che ce l'ha come inverso destro, deve essere un isomorfismo, quindi la monade è idempotente.

  Ora, la lax idempotenza si apprezza tutta nel fatto che $\clE\cap \clM$ è fatta dagli aggiunti sinistri! Ciò che otteniamo, ora, è infatti che $\eta_{RA}$ è un aggiunto sinistro, e il suo aggiunto destro è un candidato a fare da $\mu_A : RRA \to RA$.
\end{itemize}
\section{Un problema concreto: quanto è unica una fattorizzazione?}
Un problema che è passato sotto silenzio è questo: dobbiamo ancora capire come si risponde quindi ogni proposta è bene accetta, anche se sembra stupida.
\begin{itemize}
  \item Quanto commuta il triangolo in cui un LOFS fattorizza un morfismo? La domanda sembra stupida, ma considera l'inviluppo di Isbell di una categoria $A$: si tratta della categoria delle lax-fattorizzazioni di $\hom_A$ attraverso l'oggetto terminale. Quindi in almeno un contesto una fattorizzazione lax ``esiste in natura''.
  \item Quanto unica è la fattorizzazione una volta trovata? Questo apre una porta molto interessante... classicamente infatti, il trucco per mostrare che in una coppia di fattorizzazioni $A \to Ff \to B$ e $A \to Gf\to B$ i due oggetti $Ff,Gf$ sono isomorfi usa l'ortogonalità, e l'unicità dei lifting. Entrambe proprietà che qui abbiamo opportunamente laxificato.
\end{itemize}
\section{Uniqueness of fillers}
\end{document}
