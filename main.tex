\documentclass{amsart}

\usepackage{fouche}

\title{a thing about things}
\author{
    Ivan Di Liberti
  , Leonardo Larizza
  , Fosco Loregian
}

\def\lp{\mathrel{\mathpalette{\relax}{\Finv}}}
\usepackage{booktabs}

\def\A{\protect{\{\begin{smallmatrix}
  \text{monadi}\\
  \text{lax idem}
\end{smallmatrix}\}}}
\def\B{\protect{\{\begin{smallmatrix}
  \text{LOFS}\\
  \text{riflessivi}
\end{smallmatrix}\} }}
\def\C{\protect{\{\begin{smallmatrix}
  \text{subcat}\\
  \text{lax refl}
\end{smallmatrix} \} }}
\begin{document}
\maketitle
Una lista di cose da fare:
\begin{itemize}
  \item rifare i FS lax (seguendo Freyd-Kelly) e partendo dalle proprietà di base ($\clE\cap \clM = ?$ Quanto è unica una fattorizzazione?)
  \item Rifare [CHK] in versione lax: 
  \[
  \xymatrix{
    \A \ar@<5pt>[dr]^1 && \ar[ll]_{3} \B \ar@<5pt>[dl]^{2'} \\
     & \C \ar@<5pt>[ur]^2 \ar@<5pt>[ul]^{1'} &
  }  
  \]  
  Per mostrare queste varie implicazioni bisogna controllare che
  \begin{itemize}
  \item[1.] Le algebre di una KZ sono una categoria lax reflective
  \item[1'.] Una subcat lax reflective dà luogo a una monade lax idem
  \item[3.] dovrebbe già stare in qualche paper
  \item[2.] $\clB\subset\clA$ subrefl implica che $({}^\perp(\hom \clB), ({}^\perp(\hom\clB))^\perp)$ è un pre-LOFS
  \item[2'.] La subcat $\{X \mid X \to * \in \clM\}$ è lax reflective
  \end{itemize}
\end{itemize}
\hrulefill

\section{Un po' di cose da fare}
\begin{definition}[Lax e oplax ortogonalità]
  
\end{definition}
\begin{remark}
  Connessione tra le due: la lax ortogonalità è oplax ortogonalità in $\clK^\co$.
\end{remark}
\begin{definition}
  La connessione di Galois tra le ortogonalità:
  \[
  {}^{\lp}(\firstblank) \dashv (\firstblank)^{\lp}
  \]
\end{definition}
\begin{definition}
  Lax prefactorization system
\end{definition}
\begin{definition}
  Left and right generated lax prefactorization system
\end{definition}
\begin{definition}
  Lax ortogonalità in una slice categoria
\end{definition}
\begin{lemma}
  TFAE:
  \begin{itemize}
    \item $f\lp f$
    \item $f \lp All$
    \item $All \lp f$
    \item $f$ è un{\dots} iso?
  \end{itemize}
\end{lemma}
\begin{remark}
  Max e min elementi nel poset delle prefactorizations
\end{remark}
\begin{proposition}
  Proprietà di chiusura:
  % Please add the following required packages to your document preamble:
% \usepackage{booktabs}
\begin{table}[]
  \begin{tabular}{@{}cc@{}}
  \toprule
  $\clH^{\lp}$                 & ${}^{\lp}\clH$                  \\ \midrule
  contiene tutti gli iso       & contiene tutti gli iso          \\
  chiuso per composizione      & chiuso per composizione         \\
  chiuso per push              & chiuso per pull                 \\
  chiuso per retratti          & chiuso per retratti             \\
  chiuso per compo transfinite & chiuso per op-compo transfinite \\ \bottomrule
  \end{tabular}
  \end{table}
\end{proposition}
\begin{proposition}
  Unicità della fattorizzazione: quanto è unica?
\end{proposition}
\begin{proposition}
  3-per-2 e proprietà di cancellazione:
  \begin{itemize}
    \item $\clE$ è R32; $vu,u\in\clE\To v \in\clE$;
    \item $\clM$ è L32; $v, vu\in\clM\To u\in\clM$.
  \end{itemize}
  In un PFS riflessivo, $\clM$ è 3-per-2; in un PFS coriflessivo, $\clE$ è 3-per-2.
\end{proposition}
\begin{theorem}
  Se $\bfC$ è una 2-categoria con un oggetto terminale, c'è un'aggiunzione
  \[
  \Phi : Rex(\bfC)_{lax} \leftrightarrows PF(\bfC)_{lax, \tau} : \Psi  
  \]
  dove 
  \begin{itemize}
    \item $\Psi : (\clE,\clM)\mapsto \{X \mid (X\to *)\in\clM\} \in ReX(\bfC)_{lax}$
    \item $\Phi : (\bfB\subseteq \bfC) \mapsto \big({}^{\lp}(\hom\bfB), ({}^{\lp}(\hom\bfB))^{\lp}\big)$
  \end{itemize}
\end{theorem}
\end{document}